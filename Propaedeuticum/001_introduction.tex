% create the chapter introduction
\chapter{Introduction}

On of the major challenges in the field of global healthcare are antibiotic resistances. Estimates suggest that the antimicrobial resistance was directly responsible for 1.27 million deaths in 2019 \cite{who_antibiotic_resistance}. Through the widly spread practice of prescribing antibiotics the number of resistant bacteria is expected to rise in the coming years \cite{versporten2018}.

The most commonenly used antibiotics are so called $\beta$-lactam antibiotics, often called $\beta$ lactamases. These antibiotics work by interfering with the synthesis of the bacterial cell wall \cite{waxman1980}. Based on the way the catalytic mechanism of the enzymes works, the $\beta$ lactamases have been classified into four different classes A, B, C and D \cite{bush2013}. One of the first found $\beta$ lactamases was an enzyme which was called TEM-1 based on the patient first three letters of their name \cite{sutcliffe1978}. Later many more TEM-like enzymes were found and classified into the same class, today there are 250 TEM-like enzymes known. Many of these enzymes vary only in one or two mutations, which makes them sequence wise similar, but have different phenotypes when it comes to antibiotic resistance \cite{naas2017}. This makes the sequence space dense while with machine learning new insights can be gained, in so called embedding spaces.

New popular approaches in machine learning have been used to make a varity of predictions concerning protein strucutre (e.g. AlphaFold \cite{jumper2021}), protein function (e.g. ESM-3 \cite{hayes2025}) and protein classification. These new approaches allow to understand decades of data and hopefullly lead to new insights in the field of antibiotic resistances.

In this thesis the focus is on the combination of avaliable and often underutilized data like NCBI, CARD and UniProt, with the new approaches from machine learning to make predictions about the antibiotic resistance of the enzymes.